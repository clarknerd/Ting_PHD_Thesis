%!TEX root = ../paper.tex
Computing either Ambiguity or Deviation requires enumerating the entire space of permitted distributions.
One approach to approximating either measure is repeatedly sampling from, rather than enumerating the space.  
However, accurate approximations require a large number of samples, rendering this approach similarly inefficient.
In this section, we propose a faster approach to assessing the fidelity of a pattern encoding.
Specifically, we select a single representative distribution $\overline \rho_\encoding$ from the space $\Omega_\encoding$, and use $\overline \rho_\encoding$ to approximate both Ambiguity and Deviation.

%%%%%%%%%%%%%%%%%%%%%%%%%%%%%%%

\subsection{\Errorname}
\label{sec:maximumentropydistribution}

\tinysection{Maximum Entropy Distribution}
The representative distribution is chosen by applying the maximum entropy principle~\cite{DBLP:journals/tssc/Jaynes68} commonly used in pattern-based summarization~\cite{DBLP:journals/tkdd/MampaeyVT12,DBLP:journals/pvldb/GebalyAGKS14}.
That is, we select the distribution $\overline{\rho}_\encoding$ with maximum entropy:
$$\overline{\rho}_\encoding=\argmaxl_{\rho\in\Omega_\encoding}\entropy(\rho)\;\;\;\;\;\;\text{where }\entropy(\rho)=\mysuml_{\vec{q}\in\{0,1\}^n}-\rho(\vec{q})\log \rho(\vec{q})$$
The maximum entropy distribution best represents the current state of knowledge.
That is, a distribution with lower entropy assumes additional constraints derived from patterns that we do not know, while one with higher entropy violates the constraints from patterns we do know.

Maximizing an objective function belonging to the exponential family (entropy in our case) under a mixture of linear equalities/inequality constraints is a convex optimization problem~\cite{Boyd:2004:CO:993483} which guarantees a \emph{unique} solution and can be efficiently solved using the cvx toolkit~\cite{cvx,DBLP:journals/jota/ODonoghueCPB16}, and/or by \textit{iterative scaling}~\cite{DBLP:journals/tkdd/MampaeyVT12,DBLP:journals/pvldb/GebalyAGKS14}.
For naive encodings specifically, we can assume independence between each feature $X_i$.  
Under this assumption, $\overline{\rho}_\encoding$ has a closed-form representation:
\begin{equation*}
\overline{\rho}_\encoding(\vec{q})=\prod_{i}p(X_i=x_i)\;\;\;\;\;\;\text{where } \vec{q}=(x_1,\ldots,x_n)
\end{equation*} 
%
%\tinysection{\Errorname}
%the representative distribution $\overline\rho_\encoding$, 
We define \textit{\errorname} $e(\encoding)$ as the entropy difference between the representative and true distributions:
$$e(\encoding)=\entropy(\overline{\rho}_\encoding)-\entropy(\rho^*)\;\;\;\;\;\text{where }\overline{\rho}_\encoding=\argminl_{\rho\in\Omega_\encoding}-\entropy(\rho)$$

%\tinysection{Relationship to K-L Divergence}
%\Errorname is closely related to the K-L divergence from the representative distribution $\overline \rho_\encoding$ to the true distribution $\rho^*$.
%\begin{multline*}
%\mathcal{D}_{KL}(\rho^*||\overline{\rho}_\encoding)=\entropy(\rho^*,\overline{\rho}_\encoding)-\entropy(\rho^*)\\
%\text{where }\entropy(\rho^*,\overline{\rho}_\encoding)=\mysuml_{\vec{q}}-\rho^*(\vec{q})\log\overline{\rho}_\encoding(\vec{q})
%\end{multline*}
%$\entropy(\rho^*,\overline{\rho}_\encoding)$ is called the \emph{cross-entropy}. Replacing cross-entropy by entropy $\entropy(\overline{\rho}_\encoding)$, the formula becomes the same as \errorname. Though cross entropy is different from entropy in general, e.g., it is only defined when $\rho^*\ll\overline{\rho}_\encoding$, they are closely correlated. 
%However, for the specific case of naive encodings the two are equivalent.
%\begin{lemma}
%\label{equivalence}
%For any naive encoding $\encoding$, $\entropy(\rho^*,\overline{\rho}_\encoding) = \entropy(\overline{\rho}_\encoding)$
%\end{lemma}
%\begin{proof}
%With $\vec{q} = (x_1,\ldots,x_n)$ and applying Equation~\ref{eqn:naiveencodingformula}: 
%{\small
%\begin{eqnarray*}
%\sum_{\vec{q}\in L}-\rho^*(\vec{q})\log\overline{\rho}_{\encoding}(\vec{q}) &=&-\sum_{\vec{q}}p(\vec{q})\cdot \sum_{i}\log p(X_i=x_i)\\
%&=&-\sum_{i,k}\log p(X_i=k)\cdot\sum_{\vec{q}\;|\;x_i = k}p(\vec{q})\\
%&=&-\sum_{i,k}\log p(X_i=k)\cdot p(X_i=k)\\
%&=&\sum_{i}\entropy(X_i)
%\end{eqnarray*}
%}
%\noindent  The variables in a naive encoding are independent, so: $\sum_{i}\entropy(X_i) = \entropy(\bar \rho_\encoding)$, and we have the lemma.
%\end{proof}
%While we do not provide a similar proof for more general encodings, we show it experimentally in Section~\ref{sec:motivateencodingerror}.
%%%%%%%%%%%%%%%%%%%%%%%%%%%%%%%

\subsection{Practical vs Idealized Information Loss}
\label{sec:validateencodingerror}

In this section we prove that \errorname closely parallels Ambiguity. 
We define a partial order lattice over encodings and show that for any pair of encodings on which the partial order is defined, a like relationship is implied for both \errorname and Ambiguity.
I supplement the proofs given in this section with an empirical analysis relating \errorname to Deviation in Section~\ref{sec:motivateencodingerror}.

\tinysection{Containment}
%Recall that each encoding $\encoding$ induces a space of distributions $\Omega_\encoding$.
We define a partial order over encodings $\leq_\Omega$ based on \textit{containment} of their induced spaces $\Omega_\encoding$:
$$\encoding_1 \leq_\Omega \encoding_2\;\;\;\equiv\;\;\;\Omega_{\encoding_1} \subseteq \Omega_{\encoding_2}$$
That is, one encoding (i.e., $\encoding_1$) precedes another (i.e., $\encoding_2$) when all distributions admitted by the former encoding are also admitted by the latter.

%%%%%%%%%%%%%%%%%%%%%%%%%%%%%%%

\tinysection{Containment Captures \Errorname}
We first prove that the total order given by \errorname is a superset of the partial order $\leq_\Omega$. 
\begin{lemma}
\label{prop:monotone}
For any pair of encodings $\encoding_1,\encoding_2$ that induce spaces $\Omega_{\encoding_1},\Omega_{\encoding_2}$ and maximum entropy distributions $\overline{\rho}_{\encoding_1},\overline{\rho}_{\encoding_2}$ it holds that $\encoding_1 \leq_\Omega \encoding_2\to e(\encoding_1)\leq e(\encoding_2)$. 
\end{lemma}
\begin{proof} 
First we have $\Omega_{\encoding_2}\supseteq \Omega_{\encoding_1}\to\overline{\rho}_{\encoding_1}\in\Omega_{\encoding_2}$. Since $\overline{\rho}_{\encoding_2}$ has the maximum entropy among all distributions $\rho\in\Omega_{\encoding_2}$, we have $\entropy(\overline{\rho}_{\encoding_1})\leq\entropy(\overline{\rho}_{\encoding_2})\equiv e(\encoding_1)\leq e(\encoding_2)$. 
\end{proof}

%%%%%%%%%%%%%%%%%%%%%%%%%%%%%%%

\tinysection{Containment Captures Ambiguity} 
Next, we show that the partial order based on containment implies a like relationship between Ambiguities of pairs of encodings.
\begin{lemma}
\label{prop:containmentrepresentativeness}
Given encodings $\encoding_1,\encoding_2$ with uninformed prior on $\mathcal{P}_{\encoding_1},\mathcal{P}_{\encoding_2}$, it holds that $\encoding_1 \leq_\Omega \encoding_2\to \text{I}(\encoding_1)\leq \text{I}(\encoding_2)$.
\end{lemma}
\begin{proof}
Given an uninformed prior: $\text{I}(\encoding)=\log|\Omega_\encoding|$, we have $\encoding_1 \leq_\Omega \encoding_2\to|\Omega_{\encoding_1}|\leq|\Omega_{\encoding_2}|\to \text{I}(\encoding_1)\leq \text{I}(\encoding_2)$ 
\end{proof}

