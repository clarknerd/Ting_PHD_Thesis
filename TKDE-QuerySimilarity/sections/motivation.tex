%!TEX root = ./sections/1-intro.tex
\subsection{Motivating Application: \sysname{}}
\label{sec:motivation}
It is increasingly important for organizations to be able to detect and respond to cyber attacks.
An especially difficult class of cyber attack to detect is the so called \textit{insider attacks} that occur when employees misuse legitimate access to a resource like a database.
The difficulty arises because apparently anomalous behavior from a legitimate actor might still have legitimate intent.  
For example, a bank teller in Buffalo who withdraws a large sum for a client from California may be acting legitimately (e.g., if the client has just moved and is purchasing a house), or may be committing fraud.
The ``U.S. State of Cybercrime Survey''~\cite{cybercrimeReport2014} states that 37\% of organizations have experienced an insider incident, while a 2015 study~\cite{ponemonReport2015} identified insider attacks as having the longest average response time of any attack type surveyed: 54.5 days.

The challenge of addressing of insider attacks lies in the difficulty of precisely specifying access policies for shared resources such as databases. 
Coarse, permissive access policies provide opportunities for exploitation. 
Conversely, restrictive fine-grained policies are expensive to create and limit a legitimate actor's ability to adapt to new or unexpected tasks.
In practice, enterprise database system administrators regularly eschew fine-grained database-level access control.  
Instead, large companies commonly rely on reactive strategies that monitor external factors like network activity patterns and shared file transfers.
In a corporate environment, monitoring user actions requires less preparation and gives users a greater degree of flexibility. 
However, external factors do not always provide a strong attestation of the legitimacy of a database user's actions. 

The \sysname{}\footnote{\sysname{} is derived from the last words of the Roman emperor Julius Caesar, ``\textit{Et tu, Brute?}'' in Latin, meaning ``\textit{You, too, Brutus?}'' in English to emphasize that this system is meant to detect the unexpected betrayals of trusted people} system, currently under development at the University at Buffalo~\cite{kul2016ettu}, seeks to help analysts to monitor query access patterns for signs of insider attack.  Database logging and monitoring is expensive, so \sysname{} needs to be able to identify normal, baseline database behaviors that can be easily flagged as ``safe'' and ignored from normal logging and post-mortem attack analysis.  In this paper, we focus on one concrete part of the overall \sysname{} system, responsible for summarizing and visualizing query logs.  In the complete system, this component serves to help analysts generate patterns of safe queries, and to quickly analyze large multi-day query logs to identify potential attack activity.
