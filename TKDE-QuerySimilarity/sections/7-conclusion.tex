% !TEX root = ../paper.tex
The focus of this work is to understand and improve similarity metrics for SQL queries relying on query structure to be used to cluster queries. 
We described a quality evaluation scheme that captures the notion of query task using student answers to query-construction problems and a real-world smartphone query load.
We used this scheme to evaluate three existing query similarity metrics.
We also proposed a feature engineering technique for standardizing query representations.
Through further experiments, we showed that different workloads have different characteristics and no one similarity metric surveyed was always good. 
The feature engineering steps provided an improvement across the board because they addressed the error reasons we identified. 

The approaches described in this article only represent the first steps towards tools for summarizing logs by tasks.
Concretely, we plan to extend our work in several directions:
First, we will explore new feature extracting mechanisms like the Weisfeiler-Lehman framework~\cite{kul2016ettu}, feature weighting strategies and new labeling rules in order to capture the task behind logged queries better.
Second, we will introduce the temporal order of the log to increase the query clustering quality. In this article, we focused on query structures to improve clustering quality. Exploring the inter-query feature correlation based on query order can be used to summarize query logs in addition to clustering.
Third, we will examine user interfaces that better present clusters of queries --- Different feature sorting strategies in Frequent Pattern Trees (FP Trees)~\cite{han2004mining} in order to help the user distinguish important and irrelevant features, for example.
Lastly, we will investigate the temporal effects on query clustering. 
