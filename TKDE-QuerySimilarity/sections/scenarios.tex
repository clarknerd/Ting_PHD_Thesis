%!TEX root = ./sections/1-intro.tex
In this section, we provide three scenarios where the clustering scheme coupled with the proposed regularization is applicable:

The first one is, \textit{Jane the DBA} where she takes on the task of improving database performance.
After performing the straightforward database indexing tasks, she would need to select candidate \textit{views}, which are virtual tables defined by a query.
They allow querying just like tables by pre-fetching records from existing tables. 
Constructing a view for a frequent complex join operation can increase querying performance of the database substantially.
To find the ideal views, Jane first clusters similar queries together to see what kinds of queries are more frequent.
Making the most frequent complex query types faster by creating views of them could improve database performance substantially~\cite{aouiche2006, aligon2014similarity}. 

The second one is, \textit{Jane the security auditor} where she suspects that there is a person who leaks classified information from her organization.
She can choose to investigate database access patterns along with other strategies which would involve query clustering~\cite{Sun2016}.
After identifying the query clusters, she can partition the queries by the department or role to get the intuition about which departments and roles \textit{normally} utilize what part of the database.
She can detect the \textit{outliers} from that behavior in order to determine the suspects for further investigation.

Lastly, \textit{Jane the researcher} where needs to investigate the properties of the SQL query dataset that she is going to use for her research.
One of the new graduate students in her team clusters the queries, and provides her with the clustering assignments of each query.
She doubts the quality of the clustering performed, and wonders if the clustering operation could be performed better.

Having a \textit{better} clustering of queries would potentially enhance the quality of her work in all of the examples given above.
Also, works cited in this section~\cite{aouiche2006, aligon2014similarity, Sun2016}, along with many others can benefit from the framework described in this article.
