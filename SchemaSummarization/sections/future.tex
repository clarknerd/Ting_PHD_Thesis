% !TEX root = ../cnf-json.tex


One challenge that we will need to address in \systemnametwo is coping with nested collections.  
At the moment, the user can manually merge collections of attributes that correspond to disjoint entites.
However, we would like to automate this process.
One observation is that a typical collection like an array has a schema with the general structure:
$$(P_1 \vee P_1P_2 \vee P_1P_2P_3 \vee \ldots)\;=\;(P_1 \wedge (\emptyset \vee P_2 \wedge (\emptyset \vee P_3 \wedge (\ldots)) ))$$
The version of this expression on the right hand side is notable as its closure over the semiring $\tuple{\big\{\{\mathbf P\}\big\}, \vee, \wedge, \emptyset, \big\{\{\}\big\}}$ would indicate that the semiring is ``quasiregular'' or ``closed'', an algebraic structure best associated with the Kleene star.
Hence, we plan to explore the use of the Kleene star to encode nested collections in our algebra.
A key challenge in doing so is detecting opportunities for incorporating it into a summary, a more challenging form of the factorization problem.

A further step to increase the capabilities of \systemnametwo is to incorporate type information in the summarization. 
This adds an extra layer of information an analyst can extract from our system, as well as the ability to identify and correct schema errors. 
As a long term goal we will provide capabilities for linking views, for example by defining functional dependencies.
The goal is to create full entity relationship diagrams. 
In particular, one interesting way to identify potential relationships that exist between entities is by leveraging the overlap between segments.