% -*- root: ../cnf-json.tex -*-
Ad-hoc data models like \json make it easy to evolve schemas and to multiplex different data-types into a single stream.
This flexibility makes \json great for generating data, but also makes it much harder to query, ingest into a database, and index.
In this paper, we explore the first step of \json data loading: schema design.
Specifically, we consider the challenge of designing schemas for existing \json datasets as an \emph{interactive} problem.  
We present \systemnametwo, a roll-up/drill-down style interface for exploring collections of \json records.
\systemnametwo helps users to visualize the collection, identify relevant fragments, and map it down into one or more flat, relational schemas.
We describe and evaluate two key components of \systemnametwo: 
(1) A summary schema representation that significantly reduces the complexity of JSON schemas without a meaningful reduction in information content, 
and (2) A collection of schema visualizations that help users to qualitatively survey variability amongst different schemas in the collection.