%%%%%% Standard Packages %%%%%%
\usepackage{booktabs} % For formal tables
\usepackage{balance}
\usepackage{graphicx}
\usepackage[pdftex,letterpaper]{geometry} % fixed to US letter size for output (since version 1.8c - UR 2010)
\usepackage{epsfig}
\usepackage{amssymb}
\usepackage{amsmath}
\usepackage{amsfonts}
\usepackage{amsthm}
%\usepackage[noend]{algorithmic}
\usepackage{algorithm}
\usepackage{stackrel}
\usepackage{algpseudocode}
\usepackage{textcomp}
\PassOptionsToPackage{table}{xcolor}
\usepackage[table]{xcolor}
\usepackage{listings}
\usepackage{subcaption}
% \usepackage{caption}
\usepackage{mathtools,xparse}
\usepackage{dsfont}
\usepackage{stmaryrd}
% \usepackage[inline]{enumitem}

%\usepackage{cite}
\usepackage{paralist}
\usepackage{hyperref}
\hypersetup{hidelinks=true}
\usepackage{cleveref}
\usepackage{lipsum}
\usepackage[normalem]{ulem}
\usepackage{array}
\usepackage{multicol}
\usepackage{enumitem}
\usepackage{etoolbox}
\usepackage{varwidth}
\usepackage{xspace}

\usepackage{multirow}

%%%%%% Package Configuration %%%%%%
%%% Listings
\lstset{language=sql,morekeywords={LENS,SCHEMA_MATCHING,string},basicstyle=\small\upshape\ttfamily,keywordstyle=\color{blue}}
%%% verbatim
\makeatletter
\preto{\@verbatim}{\topsep=5pt \parsep=0pt }
\makeatother
%%% Algorithmic
\renewcommand{\algorithmicrequire}{\textbf{In:}}
\renewcommand{\algorithmicensure}{\textbf{Out:}}
%% Multicol
\setlength\multicolsep{\topsep}

%%%%%% Standard Theorem Environments %%%%%%
%\newtheorem{example}{Example}
\newtheorem{scenario}{Scenario}
%\newtheorem{definition}{Definition}
\newtheorem{property}{Property}
\newtheorem{transformation}{Transformation}

%%%%%%% Table styling %%%%%%
\newcolumntype{L}[1]{>{\raggedright\let\newline\\\arraybackslash\hspace{0pt}}m{#1}}
\newcolumntype{C}[1]{>{\centering\let\newline\\\arraybackslash\hspace{0pt}}m{#1}}
\newcolumntype{R}[1]{>{\raggedleft\let\newline\\\arraybackslash\hspace{0pt}}m{#1}}


%%%%%% Common Math-Mode Aliases %%%%%%
\newcommand{\comprehension}[2]{\left\{\left.\;{#1}\;\right|\;{#2}\;\right\}}
%\newcommand{\bagcomprehension}[2]{\llbrace\left.\;{#1}\;\right|\;{#2}\;\rrbrace}
\newcommand{\bagcomprehension}[2]{\left\{\!\left|\left.\;{#1}\;\right|\;{#2}\;\right|\!\right\}}
\newcommand{\setsize}[1]{\left|#1\right|}
\newcommand{\tuple}[1]{\left<\;{#1}\;\right>}
\newcommand{\ordefn}{\;|\;}
\newcommand{\sch}[1]{\texttt{schema}({#1})}
\newcommand{\projection}{\pi}
\newcommand{\selection}{\sigma}
\newcommand{\logicalAnd}{\wedge}
\newcommand{\logicalOr}{\vee}
\newcommand{\Union}{\bigcup}
\newcommand*{\Unionl}{\Union\limits}
\newcommand{\Intersection}{\bigcap}
\newcommand{\E}{\mathrm{E}}
\newcommand{\expect}{\mathbb{E}}

\newcommand{\entropy}{\mathcal{H}}

\newcommand{\distinct}[1]{\lfloor{#1}\rfloor}

\DeclareMathOperator*{\argmin}{arg\,min}
\newcommand*{\argminl}{\argmin\limits}

\DeclareMathOperator*{\argmax}{arg\,max}
\newcommand*{\argmaxl}{\argmax\limits}

\DeclareMathOperator*{\mysum}{\sum}
\newcommand*{\mysuml}{\mysum\limits}

\DeclareMathOperator*{\myPi}{\Pi}
\newcommand*{\myPil}{\myPi\limits}

\DeclareMathOperator*{\mymin}{\min}
\newcommand*{\myminl}{\mymin\limits}

\DeclareMathOperator*{\mymax}{\max}
\newcommand*{\mymaxl}{\mymax\limits}

\DeclareMathOperator*{\myunion}{\bigcup}
\newcommand*{\myunionl}{\myunion\limits}

\renewcommand{\vec}[1]{\mathbf{#1}}

\DeclarePairedDelimiter{\norm}{\lVert}{\rVert}

%%%%%% TODOs %%%%%%
\newcommand{\todo}[1]{\textcolor{red}{[[ #1 ]]}}

%%%%%% Other Aliases %%%%%%
\newcommand{\ccomment}[1]{{\small\texttt{/*} #1 \texttt{*/}}}
\newcommand{\tinysection}[1]{\smallskip \noindent \textbf{#1.}~}
%\newcommand{\keyword}[1]{\textcolor{blue}{\texttt{#1}}}

\newcommand{\systemname}{\textsc{LogR}\xspace}
\newcommand{\Errorname}{Reproduction Error\xspace}
\newcommand{\errorname}{\Errorname}%reproduction error\xspace}
\newcommand{\cqword}[2]{{\footnotesize $\tuple{\text{\lstinline{#2}}, \text{\lstinline{#1}}}$}}
\newcommand{\pattern}{\vec b}
\newcommand{\encoding}{\mathcal E}
\newcommand{\naiveencoding}{\ddot{\mathcal E}}
\definecolor{light-gray}{gray}{0.75}
\definecolor{mid-gray}{gray}{0.45}

\newcommand{\newcontent}[1]{\textcolor{red}{#1}}

% adjust how much whitespace appears after figures
\newcommand{\trimfigurewhitespace}{\vspace*{-5mm}}
% BF caption for figures
\newcommand{\bfcaption}[1]{\caption{\textbf{#1}}}

\newtheorem{example}{Example}
\newtheorem{proposition}{Proposition}
\newtheorem{lemma}{Lemma}
